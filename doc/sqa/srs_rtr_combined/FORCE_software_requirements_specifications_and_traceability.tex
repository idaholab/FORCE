%
% This is an example LaTeX file which uses the SANDreport class file.
% It shows how a SAND report should be formatted, what sections and
% elements it should contain, and how to use the SANDreport class.
% It uses the LaTeX article class, but not the strict option.
% ItINLreport uses .eps logos and files to show how pdflatex can be used
%
% Get the latest version of the class file and more at
%    http://www.cs.sandia.gov/~rolf/SANDreport
%
% This file and the SANDreport.cls file are based on information
% contained in "Guide to Preparing {SAND} Reports", Sand98-0730, edited
% by Tamara K. Locke, and the newer "Guide to Preparing SAND Reports and
% Other Communication Products", SAND2002-2068P.
% Please send corrections and suggestions for improvements to
% Rolf Riesen, Org. 9223, MS 1110, rolf@cs.sandia.gov
%
\documentclass[pdf,12pt]{../../user_manual/INLreport}
% pslatex is really old (1994).  It attempts to merge the times and mathptm packages.
% My opinion is that it produces a really bad looking math font.  So why are we using it?
% If you just want to change the text font, you should just \usepackage{times}.
% \usepackage{pslatex}
\usepackage{times}
%\usepackage{longtable}
\usepackage[FIGBOTCAP,normal,bf,tight]{subfigure}
\usepackage{amsmath}
\usepackage{tabularx}
\usepackage{ltablex}
\usepackage{amssymb}
\usepackage[labelfont=bf]{caption}
\usepackage{pifont}
\usepackage{enumerate}
\usepackage{listings}
\usepackage{fullpage}
\usepackage{xcolor}          % Using xcolor for more robust color specification
\usepackage{ifthen}          % For simple checking in newcommand blocks
\usepackage{textcomp}
%\usepackage{authblk}         % For making the author list look prettier
%\renewcommand\Authsep{,~\,}

% Custom colors
\definecolor{deepblue}{rgb}{0,0,0.5}
\definecolor{deepred}{rgb}{0.6,0,0}
\definecolor{deepgreen}{rgb}{0,0.5,0}
\definecolor{forestgreen}{RGB}{34,139,34}
\definecolor{orangered}{RGB}{239,134,64}
\definecolor{darkblue}{rgb}{0.0,0.0,0.6}
\definecolor{gray}{rgb}{0.4,0.4,0.4}

\lstset {
  basicstyle=\ttfamily,
  frame=single
}

\setcounter{secnumdepth}{5}
\lstdefinestyle{XML} {
    language=XML,
    extendedchars=true,
    breaklines=true,
    breakatwhitespace=true,
%    emph={name,dim,interactive,overwrite},
    emphstyle=\color{red},
    basicstyle=\ttfamily,
%    columns=fullflexible,
    commentstyle=\color{gray}\upshape,
    morestring=[b]",
    morecomment=[s]{<?}{?>},
    morecomment=[s][\color{forestgreen}]{<!--}{-->},
    keywordstyle=\color{cyan},
    stringstyle=\ttfamily\color{black},
    tagstyle=\color{darkblue}\bf\ttfamily,
    morekeywords={name,type},
%    morekeywords={name,attribute,source,variables,version,type,release,x,z,y,xlabel,ylabel,how,text,param1,param2,color,label},
}
\lstset{language=python,upquote=true}

\usepackage{titlesec}
\newcommand{\sectionbreak}{\clearpage}
\setcounter{secnumdepth}{4}

%\titleformat{\paragraph}
%{\normalfont\normalsize\bfseries}{\theparagraph}{1em}{}
%\titlespacing*{\paragraph}
%{0pt}{3.25ex plus 1ex minus .2ex}{1.5ex plus .2ex}

%%%%%%%% Begin comands definition to input python code into document
\usepackage[utf8]{inputenc}

% Default fixed font does not support bold face
\DeclareFixedFont{\ttb}{T1}{txtt}{bx}{n}{9} % for bold
\DeclareFixedFont{\ttm}{T1}{txtt}{m}{n}{9}  % for normal

\usepackage{listings}

% Python style for highlighting
\newcommand\pythonstyle{\lstset{
language=Python,
basicstyle=\ttm,
otherkeywords={self, none, return},             % Add keywords here
keywordstyle=\ttb\color{deepblue},
emph={MyClass,__init__},          % Custom highlighting
emphstyle=\ttb\color{deepred},    % Custom highlighting style
stringstyle=\color{deepgreen},
frame=tb,                         % Any extra options here
showstringspaces=false            %
}}


% Python environment
\lstnewenvironment{python}[1][]
{
\pythonstyle
\lstset{#1}
}
{}

% Python for external files
\newcommand\pythonexternal[2][]{{
\pythonstyle
\lstinputlisting[#1]{#2}}}

\lstnewenvironment{xml}
{}
{}

% Python for inline
\newcommand\pythoninline[1]{{\pythonstyle\lstinline!#1!}}

\def\DRAFT{} % Uncomment this if you want to see the notes people have been adding
% Comment command for developers (Should only be used under active development)
\ifdefined\DRAFT
  \newcommand{\nameLabeler}[3]{\textcolor{#2}{[[#1: #3]]}}
\else
  \newcommand{\nameLabeler}[3]{}
\fi
% Commands for making the LaTeX a bit more uniform and cleaner
\newcommand{\TODO}[1]    {\textcolor{red}{\textit{(#1)}}}
\newcommand{\xmlAttrRequired}[1] {\textcolor{red}{\textbf{\texttt{#1}}}}
\newcommand{\xmlAttr}[1] {\textcolor{cyan}{\textbf{\texttt{#1}}}}
\newcommand{\xmlNodeRequired}[1] {\textcolor{deepblue}{\textbf{\texttt{<#1>}}}}
\newcommand{\xmlNode}[1] {\textcolor{darkblue}{\textbf{\texttt{<#1>}}}}
\newcommand{\xmlString}[1] {\textcolor{black}{\textbf{\texttt{'#1'}}}}
\newcommand{\xmlDesc}[1] {\textbf{\textit{#1}}} % Maybe a misnomer, but I am
                                                % using this to detail the data
                                                % type and necessity of an XML
                                                % node or attribute,
                                                % xmlDesc = XML description
\newcommand{\default}[1]{~\\*\textit{Default: #1}}
\newcommand{\nb} {\textcolor{deepgreen}{\textbf{~Note:}}~}

%%%%%%%% End comands definition to input python code into document

%\usepackage[dvips,light,first,bottomafter]{draftcopy}
%\draftcopyName{Sample, contains no OUO}{70}
%\draftcopyName{Draft}{300}

% The bm package provides \bm for bold math fonts.  Apparently
% \boldsymbol, which I used to always use, is now considered
% obsolete.  Also, \boldsymbol doesn't even seem to work with
% the fonts used in this particular document...
\usepackage{bm}

% Define tensors to be in bold math font.
\newcommand{\tensor}[1]{{\bm{#1}}}

% Override the formatting used by \vec.  Instead of a little arrow
% over the letter, this creates a bold character.
\renewcommand{\vec}{\bm}

% Define unit vector notation.  If you don't override the
% behavior of \vec, you probably want to use the second one.
\newcommand{\unit}[1]{\hat{\bm{#1}}}
% \newcommand{\unit}[1]{\hat{#1}}

% Use this to refer to a single component of a unit vector.
\newcommand{\scalarunit}[1]{\hat{#1}}

% \toprule, \midrule, \bottomrule for tables
\usepackage{booktabs}

% \llbracket, \rrbracket
\usepackage{stmaryrd}

\usepackage{hyperref}
\hypersetup{
    colorlinks,
    citecolor=black,
    filecolor=black,
    linkcolor=black,
    urlcolor=black
}
%\usepackage[table,xcdraw]{xcolor}
\newcommand{\wiki}{\href{https://github.com/idaholab/raven/wiki}{RAVEN wiki}}

% Compress lists of citations like [33,34,35,36,37] to [33-37]
\usepackage{cite}

% If you want to relax some of the SAND98-0730 requirements, use the "relax"
% option. It adds spaces and boldface in the table of contents, and does not
% force the page layout sizes.
% e.g. \documentclass[relax,12pt]{SANDreport}
%
% You can also use the "strict" option, which applies even more of the
% SAND98-0730 guidelines. It gets rid of section numbers which are often
% useful; e.g. \documentclass[strict]{SANDreport}

% The INLreport class uses \flushbottom formatting by default (since
% it's intended to be two-sided document).  \flushbottom causes
% additional space to be inserted both before and after paragraphs so
% that no matter how much text is actually available, it fills up the
% page from top to bottom.  My feeling is that \raggedbottom looks much
% better, primarily because most people will view the report
% electronically and not in a two-sided printed format where some argue
% \raggedbottom looks worse.  If we really want to have the original
% behavior, we can comment out this line...
\raggedbottom
\setcounter{secnumdepth}{5} % show 5 levels of subsection
\setcounter{tocdepth}{5} % include 5 levels of subsection in table of contents

% ---------------------------------------------------------------------------- %
%
% Set the title, author, and date
%
\title{FORCE Software Requirements Specification and Traceability Matrix}
%\author{%
%\begin{tabular}{c} Author 1 \\ University1 \\ Mail1 \\ \\
%Author 3 \\ University3 \\ Mail3 \end{tabular} \and
%\begin{tabular}{c} Author 2 \\ University2 \\ Mail2 \\ \\
%Author 4 \\ University4 \\ Mail4\\
%\end{tabular} }


\author{Konor Frick, Paul Talbot}
 

% There is a "Printed" date on the title page of a SAND report, so
% the generic \date should [WorkingDir:]generally be empty.
\date{}


% ---------------------------------------------------------------------------- %
% Set some things we need for SAND reports. These are mandatory
%
\SANDnum{SPC-3171}
\SANDprintDate{\today}
\SANDauthor{Konor Frick, Paul Talbot}
\SANDreleaseType{Revision 0}

% ---------------------------------------------------------------------------- %
% Include the markings required for your SAND report. The default is "Unlimited
% Release". You may have to edit the file included here, or create your own
% (see the examples provided).
%
% \include{MarkOUO} % Not needed for unlimted release reports

\def\component#1{\texttt{#1}}

% ---------------------------------------------------------------------------- %
\newcommand{\systemtau}{\tensor{\tau}_{\!\text{SUPG}}}

% Added by Sonat
\usepackage{placeins}
\usepackage{array}

\newcolumntype{L}[1]{>{\raggedright\let\newline\\\arraybackslash\hspace{0pt}}m{#1}}
\newcolumntype{C}[1]{>{\centering\let\newline\\\arraybackslash\hspace{0pt}}m{#1}}
\newcolumntype{R}[1]{>{\raggedleft\let\newline\\\arraybackslash\hspace{0pt}}m{#1}}

% end added by Sonat
% ---------------------------------------------------------------------------- %
%
% Start the document
%

\begin{document}
    \maketitle

    % ------------------------------------------------------------------------ %
    % An Abstract is required for SAND reports
    %
%    \begin{abstract}
%    \input abstract
%    \end{abstract}


    % ------------------------------------------------------------------------ %
    % An Acknowledgement section is optional but important, if someone made
    % contributions or helped beyond the normal part of a work assignment.
    % Use \section* since we don't want it in the table of context
    %
%    \clearpage
%    \section*{Acknowledgment}



%	The format of this report is based on information found
%	in~\cite{Sand98-0730}.


    % ------------------------------------------------------------------------ %
    % The table of contents and list of figures and tables
    % Comment out \listoffigures and \listoftables if there are no
    % figures or tables. Make sure this starts on an odd numbered page
    %
    \cleardoublepage		% TOC needs to start on an odd page
    \tableofcontents
    %\listoffigures
    %\listoftables


    % ---------------------------------------------------------------------- %
    % An optional preface or Foreword
%    \clearpage
%    \section*{Preface}
%    \addcontentsline{toc}{section}{Preface}
%	Although muggles usually have only limited experience with
%	magic, and many even dispute its existence, it is worthwhile
%	to be open minded and explore the possibilities.


    % ---------------------------------------------------------------------- %
    % An optional executive summary
    %\clearpage
    %\section*{Summary}
    %\addcontentsline{toc}{section}{Summary}
    %\input{Summary.tex}
%	Once a certain level of mistrust and skepticism has
%	been overcome, magic finds many uses in todays science



%	and engineering. In this report we explain some of the
%	fundamental spells and instruments of magic and wizardry. We
%	then conclude with a few examples on how they can be used
%	in daily activities at national Laboratories.


    % ---------------------------------------------------------------------- %
    % An optional glossary. We don't want it to be numbered
%    \clearpage
%    \section*{Nomenclature}
%    \addcontentsline{toc}{section}{Nomenclature}
%    \begin{description}
%          \item[alohomoral]
%           spell to open locked doors and containers
%          \item[leviosa]
%           spell to levitate objects
%    \item[remembrall]
%           device to alert you that you have forgotten something
%    \item[wand]
%           device to execute spells
%    \end{description}


    % ---------------------------------------------------------------------- %
    % This is where the body of the report begins; usually with an Introduction
    %
    \SANDmain		% Start the main part of the report

\section{Introduction}
The Framework for Optimization of ResourCes and Economics is a collection of software tools,models, and datasets acquired and developed under the Integrated Energy Systems (IES) program to enable analysis of technical and economic viability of myriad IES configurations. FORCE is the consolidating interface and data repository for all the IES toolsets ranging from macrotechnoeconomic analysis to transient process modeling and experimental validation for integrated energy systems.

\\This document is aimed to report and explain the HYBRID software requirements. In addition, it reports the traceability 
matrix between software requirements and requirement tests (tests that testify the software is compliant
with respect its own requirements).

\subsection{Other Design Documentation}
Also available within the repository is the FORCE User manual within the “docs” folder. This user manual gives a detailed explanation of the installation process, system dependencies alongside links upon which where to find them, and an explanation of the use cases within the repository. 

\subsection{Dependencies and Limitations}
The software should be designed with the fewest possible constraints. 
The only primary constraint is:
\begin{enumerate}
 \item Python 3 -- \url{https://docs.conda.io/en/latest/miniconda.html}
 
 \end{enumerate} 
 
 However, enhanced capabilities will require the installation of the aforementioned plugins (HYBRID, HERON, RAVEN, TEAL) which have the dependencies shown below. 
 
 \textbf{RAVEN}
  \begin{enumerate}
 \item Visual Studio Community Edition -- Link Available on the raven github
 \item Raven specific python library set. -- Available through the install process. 
 \end{enumerate} 
 
 \textbf{HERON, TEAL}
 \begin{enumerate}
 \item Risk Analysis and Virtual ENviroment (RAVEN) -- \url{https://raven.inl.gov/SitePages/Software%20Infrastructure.aspx}
 \end{enumerate} 
  
  \textbf{HYBRID}
 \begin{enumerate}
 \item Commercial Modelica platform Dymola -- \url{https://www.3ds.com/products-services/catia/products/dymola/latest-release/}
   \end{enumerate}
  
 

\section{References}

\begin{itemize}

  \item ASME NQA 1 2008 with the NQA-1a-2009 addenda, ``Quality Assurance Requirements for Nuclear Facility Applications,'' First Edition, August 31, 2009.
  \item ISO/IEC/IEEE 24765:2010(E), ``Systems and software engineering Vocabulary,'' First Edition, December 15, 2010.
  \item LWP 13620, ``Managing Information Technology Assets''
\end{itemize}


\section{Definitions and Acronyms}

\subsection{Definitions}
\begin{itemize}
  \item \textbf{Baseline.} A specification or product (e.g., project plan, maintenance and operations [M\&O] plan, requirements, or 
design) that has been formally reviewed and agreed upon, that thereafter serves as the basis for use and further 
development, and that can be changed only by using an approved change control process. [ASME NQA-1-2008 with the 
NQA-1a-2009 addenda edited]
  \item \textbf{Validation.} Confirmation, through the provision of objective evidence (e.g., acceptance test), that the requirements 
for a specific intended use or application have been fulfilled. [ISO/IEC/IEEE 24765:2010(E) edited]
  \item \textbf{Verification.}
  \begin{itemize}
     \item The process of evaluating a system or component to determine whether the products of a given development 
     phase satisfy the conditions imposed at the start of that phase.
     \item  Formal proof of program correctness (e.g., requirements, design, implementation reviews, system tests). 
     [ISO/IEC/IEEE 24765:2010(E) edited]
  \end{itemize}
\end{itemize}

\subsection{Acronyms}
\begin{description}
\item[API] Application Programming Interfaces
\item[ANL] Argonne National Laboratory
\item[ARMA] Auto-Regressive Moving Average
\item[DOE] Department of Energy
\item[FMI] Functional Mock-up Interface
\item[FMU] Functional Mock-up Unit
\item[HERON] Heuristic Energy Resource Optimization Network
\item[IES] Integrated Energy Systems
\item[INL] Idaho National Laboratory
\item[NHES] Nuclear-Renewable Hybrid Energy Systems 
\item[IT] Information Technology
\item[ORNL] Oak Ridge National Laboratory
\item[M\&O] Maintenance and Operations
\item[NQA] Nuclear Quality Assurance
\item[POSIX]  Portable Operating System Interface
\item[QA] Quality Assurance
\item[RAVEN] Risk Analysis and Virtual ENviroment
\item[SDD] System Design Description
\item[TEAL] Tool for Economic Analysis
\item[TRANSFORM]  Transient Simulation Framework of Reconfigurable Modules
\item[XML] eXtensible Markup Language 
\end{description}

%\input{../srs/requirements.tex}
\subsection{System Operations}
\subsubsection{Human System Integration Requirements}
The command line interface shall support the ability to toggle any supported coloring schemes on or off pursuant to section 
508 of the Rehabilitation Act of 1973.
\subsubsection{Maintainability}
\begin{itemize}
  \item The latest working version (defined as the version that passes all tests in the current regression test suite) shall be 
           publicly available at all times through the repository host provider.
  \item  Flaws identified in the system shall be reported and tracked in a ticket or issue based system. The technical lead or 
            any COB member will 
            determine the severity and priority of all reported issues. The technical lead will assign resources at his or her 
            discretion to resolve identified issues.
  \item  The software maintainers will entertain all proposed changes to the system in a timely manner 
           (within two business days).        
  \item  The FORCE framework in its entirety is made publicly available under the Apache version 2.0 license.
\end{itemize}
\subsubsection{Human System Integration Requirements}
The regression test suite will cover at least 80\% of all models at all times.
The results of the regression tests will be stored in the Continuous Integration System.

\subsection{Information Management}
The FORCE framework in its entirety is made publicly available on an appropriate repository hosting site (e.g. GitHub).
Backups and security services will be provided by the hosting service.

\section{Verification}
The regression test suite shall employ several verification tests of the correct mechanical executions
of the models and workflows reported in this repository.

 \section{FORCE:SYSTEM REQUIREMENTS} 
 \subsection{Requirements Traceability Matrix} 
 This section contains all of the requirements, requirements' description, and 
 requirement test cases. The requirement tests are automatically tested for each 
 CR (Change Request) by the CIS (Continuous Integration System). 
 \newcolumntype{b}{X} 
 \newcolumntype{s}{>{\hsize=.5\hsize}X} 
 \subsubsection{Minimum Requirements} 
\begin{tabularx}{\textwidth}{|s|s|b|} 
\hline 
\textbf{Requirment ID} & \textbf{Requirment Description} & \textbf{Test(s)}  \\ \hline 
\hline 
 \hspace{0pt}R-M-1 & \hspace{0pt}Dymola 2020x or higher & \hspace{0pt}1)K. Frick, A. Alfonsi, C. Rabiti, ``HYBRID User Manual'', INL/MIS-20-60624 \\ \hline 
\hline 
 \hspace{0pt}R-M-2 & \hspace{0pt}Visual Studio 2017 or higher with associated 64-bit Intel Compiler & \hspace{0pt}1)K. Frick, A. Alfonsi, C. Rabiti, ``HYBRID User Manual'', INL/MIS-20-60624 \\ \hline 
\hline 
 \hspace{0pt}R-M-3 & \hspace{0pt}Python 3 or higher to be able to execute RAVEN-based workflows & \hspace{0pt}1)K. Frick, A. Alfonsi, C. Rabiti, ``HYBRID User Manual'', INL/MIS-20-60624 \\ \hline 
\hline 
\caption*{Minimum Requirements}
\end{tabularx} 

\end{document}
 

    % ---------------------------------------------------------------------- %
    % References
    %

%\addcontentsline{toc}{section}{References}
%\bibliographystyle{ieeetr}
%\bibliography{raven_software_requirements_specifications}

\section*{Document Version Information}

\input{../../version.tex}

\end{document}
