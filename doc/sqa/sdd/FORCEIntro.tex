\section{Introduction}

The Framework for Optimization of ResourCes and Economics is a collection of software tools,models, and datasets acquired and developed under the Integrated Energy Systems (IES) program to enable analysis of technical and economic viability of myriad IES configurations. FORCE is the consolidating interface and data repository for all the IES toolsets ranging from macrotechnoeconomic analysis to transient process modeling and experimental validation for integrated energy systems.

To accomplish this FORCE creates automated submodules and workflows for using the additional IES opensource toolsets:
\begin{itemize}
\item HYBRID -- (\url{https://github.com/idaholab/hybrid})
\item HERON -- (\url{https://github.com/idaholab/heron})
\item TEAL -- (\url{https://github.com/idaholab/teal})
\item RAVEN -- (\url{https://github.com/idaholab/raven})
\end{itemize}

Each of which has its own use case purpose. 



\subsection{User Characteristics}

The users of the FORCE software are expected to be part of any of the following categories:
\begin{itemize}
  \item \textbf{Core developers (FORCE core team)}: These are the developers of the FORCE software or its submodules. They will be responsible for following and enforcing the appropriate software development standards. They will be responsible for designing, implementing, and maintaining the software.
  
  \item \textbf{External developers}: A Scientist or Engineer that utilizes the FORCE software and wants to extend its capabilities (new use cases, new workflow generation, etc).This user will typically have a background in modeling and simulation techniques but may only have a limited skill-set when it comes to repository structure, regression testing, and version control.
  
  \item \textbf{Analysts}:  These are users that will run the code and perform various analysis on the simulations they perform. These users may interact with developers of the system requesting new features and reporting bugs found and will typically make heavy use of the input file format.
\end{itemize}
