
\subsection{Other Design Documentation}
Also available within the repository is the FORCE User manual within the “docs” folder. This user manual gives a detailed explanation of the installation process, system dependencies alongside links upon which where to find them, and an explanation of the use cases within the repository. 

\subsection{Dependencies and Limitations}
The software should be designed with the fewest possible constraints. 
The only primary constraint is:
\begin{enumerate}
 \item Python 3 -- \url{https://docs.conda.io/en/latest/miniconda.html}
 
 \end{enumerate} 
 
 However, enhanced capabilities will require the installation of the aforementioned plugins (HYBRID, HERON, RAVEN, TEAL) which have the dependencies shown below. 
 
 \textbf{RAVEN}
  \begin{enumerate}
 \item Visual Studio Community Edition -- Link Available on the raven github
 \item Raven specific python library set. -- Available through the install process. 
 \end{enumerate} 
 
 \textbf{HERON, TEAL}
 \begin{enumerate}
 \item Risk Analysis and Virtual ENviroment (RAVEN) -- \url{https://raven.inl.gov/SitePages/Software%20Infrastructure.aspx}
 \end{enumerate} 
  
  \textbf{HYBRID}
 \begin{enumerate}
 \item Commercial Modelica platform Dymola -- \url{https://www.3ds.com/products-services/catia/products/dymola/latest-release/}
   \end{enumerate}
  
 

\section{References}

\begin{itemize}

  \item ASME NQA 1 2008 with the NQA-1a-2009 addenda, ``Quality Assurance Requirements for Nuclear Facility Applications,'' First Edition, August 31, 2009.
  \item ISO/IEC/IEEE 24765:2010(E), ``Systems and software engineering Vocabulary,'' First Edition, December 15, 2010.
  \item LWP 13620, ``Managing Information Technology Assets''
\end{itemize}


\section{Definitions and Acronyms}

\subsection{Definitions}
\begin{itemize}
  \item \textbf{Baseline.} A specification or product (e.g., project plan, maintenance and operations [M\&O] plan, requirements, or 
design) that has been formally reviewed and agreed upon, that thereafter serves as the basis for use and further 
development, and that can be changed only by using an approved change control process. [ASME NQA-1-2008 with the 
NQA-1a-2009 addenda edited]
  \item \textbf{Validation.} Confirmation, through the provision of objective evidence (e.g., acceptance test), that the requirements 
for a specific intended use or application have been fulfilled. [ISO/IEC/IEEE 24765:2010(E) edited]
  \item \textbf{Verification.}
  \begin{itemize}
     \item The process of evaluating a system or component to determine whether the products of a given development 
     phase satisfy the conditions imposed at the start of that phase.
     \item  Formal proof of program correctness (e.g., requirements, design, implementation reviews, system tests). 
     [ISO/IEC/IEEE 24765:2010(E) edited]
  \end{itemize}
\end{itemize}

\subsection{Acronyms}
\begin{description}
\item[API] Application Programming Interfaces
\item[ANL] Argonne National Laboratory
\item[ARMA] Auto-Regressive Moving Average
\item[DOE] Department of Energy
\item[FMI] Functional Mock-up Interface
\item[FMU] Functional Mock-up Unit
\item[HERON] Heuristic Energy Resource Optimization Network
\item[IES] Integrated Energy Systems
\item[INL] Idaho National Laboratory
\item[NHES] Nuclear-Renewable Hybrid Energy Systems 
\item[IT] Information Technology
\item[ORNL] Oak Ridge National Laboratory
\item[M\&O] Maintenance and Operations
\item[NQA] Nuclear Quality Assurance
\item[POSIX]  Portable Operating System Interface
\item[QA] Quality Assurance
\item[RAVEN] Risk Analysis and Virtual ENviroment
\item[SDD] System Design Description
\item[TEAL] Tool for Economic Analysis
\item[TRANSFORM]  Transient Simulation Framework of Reconfigurable Modules
\item[XML] eXtensible Markup Language 
\end{description}
